

\title{
02157 Functional programming \\
Mandatory assignment 3 \\
Solver for Propositional Logic
}
\author{
        Karol Dzitkowski - s142246
}
\date{\today}

\documentclass[12pt]{article}
\usepackage[utf8]{inputenc}

\begin{document}
\maketitle

\section{Knight-Knave puzzle}
\subsection{Formulation}
Three of the island's inhabitants – A, B, and C – were talking together. 
A said, ,,All of us are knaves.'' Then B remarked, ,,Exactly one of us is a knight.'' 
What are A, B, and C?
\subsection{Solution}
Let p, q, r be the statements that A is a knight, B is a knight and C is a knight respectively
Then $\neg p$ represents the proposition that A is a knave and $\neg q$ that B is a knave, and the same for C: $\neg r$.
We have two sentences:
\begin{enumerate}
\item $ s_1 = p \iff (\neg p \land \neg q  \land \neg r) $
\item $ s_2 = q \iff (p  \land \neg q  \land \neg r) \vee (\neg p  \land q  \land \neg r) \vee (\neg p  \land \neg q  \land r) $
\end{enumerate}
We can present a solution for that problem as: $ s_1 \land s_2$ which (after transformations) is equal to: $ q  \land \neg p \land \neg r $
Which I computed using ,,toDNFsets'' function I wrote and is included in source code. So A and C are knaves and B is a knight.
Normally this puzzle is easy because someone saying sentence 1 must be a knave, and then B must be a knight because A is not, and
if it was C than B have to be also a knight which is contradictory to sentence 2. 
\end{document}
  